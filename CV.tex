\documentclass[12pt]{article}
\usepackage[a4paper, total={19cm,27cm}]{geometry}
\usepackage{titlesec}
\usepackage{graphicx}
\usepackage{float}
\usepackage{multicol}
\setlength{\columnsep}{-9cm}
\PassOptionsToPackage{hyphens}{url}\usepackage{hyperref}
%\usepackage{fancyhdr}
%\setlength{\headheight}{15pt}
\pagestyle{empty}
%\fancyhead[L]{110-1 Bio-MEMS Fabrication Homework}
%\fancyhead[R]{鄭泊聲 b07611002}
\usepackage{amsmath}
\usepackage{latexsym}
\usepackage{multirow}
\graphicspath{ {./images/} }
%\usepackage[backend=biber, citestyle=numeric, bibstyle=numeric, sorting=none]{biblatex}
%\addbibresource{ref.bib}
\usepackage{mathspec}   %加這個就可以設定字體
\usepackage{xeCJK}       %讓中英文字體分開設置
% \setCJKmainfont{Noto Serif CJK TC} %設定中文為系統上的字型
% \newCJKfontfamily[chineseSans]\CJKsans{Noto Sans CJK TC}
\setmainfont{Sabon LT Pro}
%\setmainfont{New York Small}
%\setsansfont{Univers LT Std}
\setsansfont{SF Pro Text}
%\setmonofont{Roboto Mono}
% \XeTeXlinebreaklocale "zh"             %這兩行一定要加,中文才能自動換行
% \XeTeXlinebreakskip = 0pt plus 1pt     %這兩行一定要加,中文才能自動換行
\renewcommand{\baselinestretch}{1}
\renewcommand{\figurename}{圖}
\renewcommand{\tablename}{表}
\renewcommand{\abstractname}{摘要}
\renewcommand{\contentsname}{目錄}
\renewcommand{\listtablename}{表格目錄}
%\renewcommand*{\bibfont}{\footnotesize}
\titleformat*{\section}{\Large \bfseries}
\titleformat*{\subsection}{\large \sffamily}
\titleformat*{\subsubsection}{\bfseries}
%\setcounter{tocdepth}{2}

\begin{document}
\begin{multicols}{2}
    \section*{Po-Sheng Cheng}
    \href{https://bencer3283.github.io/art/}{\underline{Online Portfolio Link}}
    
    \columnbreak
    {\sffamily \noindent
    %\Pursuing Master of Science in Graduate Insitute of Biomedical Electronics and Bioinformatics, College of EECS. \newline
    Master of Industrial Design, Rhode Island School of Design, Jun. 2026 \newline
    B.Sc. Bio-Mechatronics Engineering \& \newline B.A Economics, National Taiwan University (NTU), Jan 2023
    }
\end{multicols}
    %\section*{Publication}
        \subsection*{E-Lab Research Assistant, RISD Industrial Design, Feb. 2024 - present}
        {\sffamily
        \begin{itemize}
            \item I hosted workshops in several introductory electronics topics like soldering, basic circuit design and Arduino/ESP32 etc.
            I also work with course teachers to host specific workshops that tailors to their course needs.
            \item I manage the inventory of various components in the E-Lab.
        \end{itemize}
        }
        \subsection*{TeleSHift: Telexisting TUI for Physical Collaboration \& Interaction}
        {\sffamily
        {\footnotesize Andrew Chen, Tzu-Ling Yang, Shu-Yan Cheng, \textbf{Po-Sheng Cheng}, Tzu-Han Lin, and Kaiyuan Lin. 2022. This work recieved the \textbf{Best Demo Award at Ubicomp/ISWC 2022 Conference}%, September 11-15, 2022, Cambridge, United Kingdom. ACM, New York, NY, USA, 4 pages. 
        , https://doi.org/10.1145/3544793.3560323}
        %\vspace*{4pt} \newline Human computer interaction
        \begin{itemize}
            \item In this work, I designed a shape-transforming device called TeleSHift with a 3D tangible user interface (TUI) for group-based collaboration.
            \item My contribution were mechanical design with extensive CAD (SolidWorks), iterative improvements to design for manufacturability (reducing assembling time of the prototype to 1/5) and FDM rapid-prototyping as well as the control circuit.
        \end{itemize}
        }
    %\section*{Experiences
    % \footnote{\sffamily My coding experiences: Java for data structure, Matlab/R for statistics, C++/Kotlin for app development, Python for data processing on Raspberry Pi, Dart for UI, Labview for system integration and data visualization.}
    %}
    % {\sffamily UI/UX, Econometrics, Machanics Design, Mechatronics, Electronics, Software Development. 
    % }
    %\begin{multicols*}{2}
        
        \subsection*{Electro-Mechanical Engineering Intern, Logitech, Feb. - Jun. 2022}
        {\sffamily
        \begin{itemize}
            \item I interviewed 50 students to understand their needs for gaming keyboard and identified an opportunity for innovation then designed three prototypes to demonstrate it.
            \item My prototyping works mainly includes mechanics design with Creo, SLA rapid-prototyping and electronic circuit as well as constantly engaging with the New Product Introduction teams across different departments in the company.
        \end{itemize}
        }
        \subsection*{College Student Researcher, Photonics Workshop, NTU, Jul. 2021 - Feb. 2022}
        {\sffamily
        \begin{itemize}
            \item I developed a novel spectral mapping system called HSI that vastly acclerate the reserch workflow of spectral measurement. (Ex. up to 26\% reduction in scanning time.)
            \item By being involved in several research projects in the institute, I was able to understand the needs and bottleneck of existing workflow and to develop a one-stop, integrated solution. 
            \item Techincal aspect: software dev., image processing and system integration. 
            \item I was awarded the \textbf{2021 Technology Innovation Award} by NTU and \textbf{College Student Research Creativity Award} by National Science and Technology Council of Taiwan with this product, HSI.
        \end{itemize}}
        \subsection*{Project Lead, Bio-Electromagnetics Laboratory, NTU, May. 2020 - Jul. 2022}
        {\sffamily
        \begin{itemize}
            \item I designed an IoT machine to monitor the amount of bugs in farm fields including its PCB and mechanics.
            \item I performed extensive CAD design for the mechanics with SolidWorks. I created a git-based collaboration workflow for SolidWorks and a Notion-based BOM managment system to help me track quotes from vendors as well as the changes in design.
        \end{itemize}
        }
        %\section*{Competitions}
        \subsection*{Golden Medalist, 19th Mobileheroes Award, Sep. - Dec. 2020}
        {\sffamily
        {\footnotesize Category of 5G Application, awarded USD\$10000 by Industrial Development Bureau of Taiwan.}
        \begin{itemize}
            \item Our team ARGO developed a AR platform that utilizes advanced image-based spatial recognition algorithm which enables real-time AR interactions on personal mobile devices. My main contribution was UI evaluation and design of the AR world for demo.   
        \end{itemize}
        }
        
\end{document}

