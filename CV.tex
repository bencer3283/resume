\documentclass[12pt]{article}
\usepackage[a4paper, total={19cm,27cm}]{geometry}
\usepackage{titlesec}
\usepackage{graphicx}
\usepackage{float}
\usepackage{multicol}
\setlength{\columnsep}{-9cm}
\PassOptionsToPackage{hyphens}{url}\usepackage{hyperref}
%\usepackage{fancyhdr}
%\setlength{\headheight}{15pt}
\pagestyle{empty}
%\fancyhead[L]{110-1 Bio-MEMS Fabrication Homework}
%\fancyhead[R]{鄭泊聲 b07611002}
\usepackage{amsmath}
\usepackage{latexsym}
\usepackage{multirow}
\graphicspath{ {./images/} }
%\usepackage[backend=biber, citestyle=numeric, bibstyle=numeric, sorting=none]{biblatex}
%\addbibresource{ref.bib}
\usepackage{mathspec}   %加這個就可以設定字體
\usepackage{xeCJK}       %讓中英文字體分開設置
% \setCJKmainfont{Noto Serif CJK TC} %設定中文為系統上的字型
% \newCJKfontfamily[chineseSans]\CJKsans{Noto Sans CJK TC}
\setmainfont{Sabon LT Pro}
\setsansfont{SF Pro Text}%Univers LT Std 55}
% \setmonofont{Roboto Mono}
% \XeTeXlinebreaklocale "zh"             %這兩行一定要加,中文才能自動換行
% \XeTeXlinebreakskip = 0pt plus 1pt     %這兩行一定要加,中文才能自動換行
% \renewcommand{\baselinestretch}{1}
% \renewcommand{\figurename}{圖}
% \renewcommand{\tablename}{表}
% \renewcommand{\abstractname}{摘要}
% \renewcommand{\contentsname}{目錄}
% \renewcommand{\listtablename}{表格目錄}
%\renewcommand*{\bibfont}{\footnotesize}
\titleformat*{\section}{\Large \bfseries}
\titleformat*{\subsection}{\large \sffamily}
\titleformat*{\subsubsection}{\bfseries}
%\setcounter{tocdepth}{2}

\begin{document}
\begin{multicols}{2}
    \section*{Po-Sheng Cheng}
    \href{https://bencer3283.github.io/art/}{\underline{Online Portfolio Link}}
    
    \columnbreak
    {\sffamily \noindent
    %\Pursuing Master of Science in Graduate Insitute of Biomedical Electronics and Bioinformatics, College of EECS. \newline
    Master of Industrial Design, Rhode Island School of Design, Jun. 2026 \newline
    B.Sc. Bio-Mechatronics Engineering \& \newline B.A Economics, National Taiwan University (NTU), Jan 2023
    }
\end{multicols}
    %\section*{Publication}
    \subsection*{TeleSHift: Telexisting TUI for Physical Collaboration \& Interaction}
    {\sffamily
    {\footnotesize Andrew Chen, Tzu-Ling Yang, Shu-Yan Cheng, \textbf{Po-Sheng Cheng}, Tzu-Han Lin, and Kaiyuan Lin. 2022. This work recieved the \textbf{Best Demo Award at Ubicomp/ISWC 2022 Conference}%, September 11-15, 2022, Cambridge, United Kingdom. ACM, New York, NY, USA, 4 pages. 
    , https://doi.org/10.1145/3544793.3560323}
        %\vspace*{4pt} \newline Human computer interaction
        \begin{itemize}
            \item In this work, I designed a shape-transforming device called TeleSHift with a 3D tangible user interface (TUI) for group-based collaboration.
            \item I laid out the system architecture of the presented prototype in this work with ESP32 microcontroller, potentiometers, DC motors, etc. I also implemented the mechanical design (SolidWorks) and power supply circuit of the prototype.
            \item Additionally, my design for manufacturability (DFM) improvements helped reducing assembling time of the prototype to only 1/5.
        \end{itemize}
        }
    %\section*{Experiences
    % \footnote{\sffamily My coding experiences: Java for data structure, Matlab/R for statistics, C++/Kotlin for app development, Python for data processing on Raspberry Pi, Dart for UI, Labview for system integration and data visualization.}
    %}
    % {\sffamily UI/UX, Econometrics, Machanics Design, Mechatronics, Electronics, Software Development. 
    % }
    %\begin{multicols*}{2}
        \subsection*{Electro-mechanical Engineering Intern, Logitech, Feb. - Jun. 2022}
        {\sffamily
        %UX survey, mechanics design
        \begin{itemize}
            \item I interviewed 50 students to understand their needs for gaming keyboard and identified an opportunity for innovation then designed three prototypes to demonstrate it.
            \item I designed a microcontroller circuit to control a stepper motor and a self-designed electromagnet. Besides, my skills in statistics with R for the UX survey, modeling with Creo and SLA/FDM prototyping were also demonstrated.
            \item I gained strong familiarity with NPI (New Product Introduction) process in the tech industry while collaborating with many departments (PM, EE, ID) in the company.
        \end{itemize}
        }
        \subsection*{College Student Researcher, NTU, Jul. 2021 - Feb. 2022}
        {\sffamily
        %Software development, system integration
        \begin{itemize}
            \item I developed a novel spectral mapping system that integrates several electrical/optical components like imaging sensors (EMCCD) with LabVIEW and C++. %\newline \href{https://bencer3283.github.io/experiences/collegeStudentResearch/}{\underline{Details linked here.}}
            \item By researching the readout sequence of the EMCCD used in the system and optimizing the algorithm, I reduced the scanning time by 26\%.
            \item By being involved in several research projects in the institute, I was able to understand the needs and bottleneck of existing workflow and to develop a one-stop, integrated solution.
            \item I was awarded the \textbf{2021 Technology Innovation Award} by CCMS, NTU and \textbf{College Student Research Creativity Award} by National Science and Technology Council of Taiwan (USD\$660) with this project.
        \end{itemize}}
        \subsection*{Project Lead, Bio-Electromagnetics Laboratory, NTU, May. 2020 - Jul. 2022}
        {\sffamily
        %Electrical system integration, mechanics design
        \begin{itemize}
            \item I designed an IoT machine to monitor the amount of bugs in farm fields including its microcontroller PCB and mechanics.
            \item I created a git-based collaboration workflow for SolidWorks and a Notion-based BOM managment system to help me track quotes from vendors as well as the changes in design.
            \item Technical aspect involved automation, IoT with Arduino (XBee), PCB design (Altium), SolidWorks, MySQL.
        \end{itemize}
        }
        
        
\end{document}

