\documentclass[12pt]{article}
\usepackage[a4paper, total={19cm,27cm}]{geometry}
\usepackage{titlesec}
\usepackage{graphicx}
\usepackage{float}
\usepackage{multicol}
\setlength{\columnsep}{-9cm}
\PassOptionsToPackage{hyphens}{url}\usepackage{hyperref}
%\usepackage{fancyhdr}
%\setlength{\headheight}{15pt}
\pagestyle{empty}
%\fancyhead[L]{110-1 Bio-MEMS Fabrication Homework}
%\fancyhead[R]{鄭泊聲 b07611002}
\usepackage{amsmath}
\usepackage{latexsym}
\usepackage{multirow}
\graphicspath{ {./images/} }
%\usepackage[backend=biber, citestyle=numeric, bibstyle=numeric, sorting=none]{biblatex}
%\addbibresource{ref.bib}
\usepackage{mathspec}   %加這個就可以設定字體
\usepackage{xeCJK}       %讓中英文字體分開設置
% \setCJKmainfont{Noto Serif CJK TC} %設定中文為系統上的字型
% \newCJKfontfamily[chineseSans]\CJKsans{Noto Sans CJK TC}
\setmainfont{Sabon LT Pro} 
\setsansfont{SF Pro Text}%Univers LT Std 55}
% \setmonofont{Roboto Mono}
% \XeTeXlinebreaklocale "zh"             %這兩行一定要加,中文才能自動換行
% \XeTeXlinebreakskip = 0pt plus 1pt     %這兩行一定要加,中文才能自動換行
\renewcommand{\baselinestretch}{0.9}
% \renewcommand{\figurename}{圖}
% \renewcommand{\tablename}{表}
% \renewcommand{\abstractname}{摘要}
% \renewcommand{\contentsname}{目錄}
% \renewcommand{\listtablename}{表格目錄}
%\renewcommand*{\bibfont}{\footnotesize}
\titleformat*{\section}{\Large \bfseries}
\titleformat*{\subsection}{\sffamily}%\large \sffamily}
\titleformat*{\subsubsection}{\bfseries}
%\setcounter{tocdepth}{2}

\begin{document}
\begin{multicols}{2}
    \section*{Po-Sheng Cheng}
    401-432-5576 \newline
    bencer@outlook.com \newline
    \href{https://bencer3283.github.io/art/}{\underline{Online Portfolio Link}}
    
    \columnbreak
    {\sffamily \small \noindent
    %\Pursuing Master of Science in Graduate Insitute of Biomedical Electronics and Bioinformatics, College of EECS. \newline
    Master of Industrial Design, Rhode Island School of Design (RISD), Jun. 2026 \newline \newline
    B.Sc. Bio-Mechatronics Engineering \& B.A Economics,\newline National Taiwan University (NTU), Jan 2023
    }
\end{multicols}
%{\small \noindent I am an aspiring prototype engineer with 5 years of experiences across a wide range of products.}
\subsection*{Notable Skills}
{ \small \begin{itemize}
    \item Electronics: STM32, Arduino, ESP32, Raspberry Pi, system interface (I2C, SPI, UART), PCB design.
    \item Mechanical: CAD (SolidWorks, Fusion and Pro/E Creo), rapid prototyping (FDM/SLA 3D Printing)
    \item Coding: C++, Java, JavaScript (Express, React), Dart (Flutter), Python (PyTorch, NumPy), Git
    \item Concept generation: Rhino, Keyshot, Figma (Wireframing), Adobe Suite
\end{itemize}}
    \section*{Experiences}
    % \subsection*{Graduate Instructor, RISD Co-Works Lab, Sep. 2024 -  \hfill Providence, RI}
    %     {\sffamily
    %     %UX survey, mechanics design
    %     \begin{itemize}
    %         \item 
    %     \end{itemize}
    %     }
    \subsection*{Apprentice, Google Hardware Product Sprint, Jun. - Sep. 2024\hfill Taipei, Taiwan}
        { \small
        %UX survey, mechanics design
        \begin{itemize}
            \item Led a team of 5 to design a Hardware as a Service platform for managing lost items in public facilities. Laid out the system architecture for both the hardware kiosk and the web services.
            \item Drove the development direction from a single product to a platform system by showcasing several prototypes built with microcontrollers incombination with mobile app and HTTP web services.
        \end{itemize}
        }
    \subsection*{E-Lab Research Assistant, RISD Industrial Design, Feb. 2024 -  \hfill Providence, RI}
        { \small
        \begin{itemize}
            \item Fostered literacy of technology in RISD by working with professors and department leaders to design devices and curriculums in topics like electronics circuit, Arduino/ESP32, RESTful API and web backend development and mobile app development.
        \end{itemize}
        }
    \subsection*{(Publication) TeleSHift: Telexisting TUI for Physical Collaboration \& Interaction}
    { \small
    {\scriptsize Andrew Chen, Tzu-Ling Yang, Shu-Yan Cheng, \textbf{Po-Sheng Cheng}, Tzu-Han Lin, and Kaiyuan Lin. 2022. This work recieved the \textbf{Best Demo Award at Ubicomp/ISWC 2022 Conference}%, September 11-15, 2022, Cambridge, United Kingdom. ACM, New York, NY, USA, 4 pages. 
    , https://doi.org/10.1145/3544793.3560323}
        %\vspace*{4pt} \newline Human computer interaction
        \begin{itemize}
            \item In this research, my implementation of a shape-transforming device with a 3D tangible user interface (TUI) was the key that enabled this work to be published and awarded at ACM Ubicomp.
            \item Laid out the system architecture of the device with ESP32 microcontroller, potentiometers, DC motors, etc. I also implemented the mechanical design (SolidWorks) and power supply circuit of the prototype.
            \item My design for manufacturability (DFM) improvements slashed assembling time by 80\% compare to the previous generation.
        \end{itemize}
        }
    %\section*{Experiences
    % \footnote{\sffamily My coding experiences: Java for data structure, Matlab/R for statistics, C++/Kotlin for app development, Python for data processing on Raspberry Pi, Dart for UI, Labview for system integration and data visualization.}
    %}
    % {\sffamily UI/UX, Econometrics, Machanics Design, Mechatronics, Electronics, Software Development. 
    % }
    %\begin{multicols*}{2}
    
        \subsection*{Electro-mechanical Engineering Intern, Logitech, Feb. - Jun. 2022\hfill Hsinchu, Taiwan}
        { \small
        %UX survey, mechanics design
        \begin{itemize}
            \item Created a brand new breed of keyboard technology by conducting robust user research with 50 interviewees and innovating a microcontroller-based prototype to control actuators like stepper motor and electromagnet. Also lots of CAD (Creo) and SLA/FDM prototyping on mechanical design. 
            \item The technology recieved widespread praise by many departments (pm, engineering, design) in the company while I gained strong familiarity with the NPI (New Product Introduction) process in the tech industry. 
        \end{itemize}
        }
        \subsection*{College Student Researcher, NTU, Jul. 2021 - Feb. 2022\hfill Taipei, Taiwan}
        { \small
        %Software development, system integration
        \begin{itemize}
            \item Developed a novel, integrated spectral mapping system that vastly accelerates scientific spectral measurement by being involved in research projects in the institution to understand the bottleneck of existing workflow.
            %integrates several electrical/optical components like imaging sensors (EMCCD) with LabVIEW and C++. %\newline \href{https://bencer3283.github.io/experiences/collegeStudentResearch/}{\underline{Details linked here.}}
            \item By using linear dispersion optics and optimizing the EMCCD readout algorithm in LabVIEW and C++, the scanning time was reduced by 26\%.
            \item Awarded the \textbf{2021 Technology Innovation Award} by CCMS, NTU and \textbf{College Student Research Creativity Award} by National Science and Technology Council of Taiwan with this project.
        \end{itemize}}
        \subsection*{Project Lead, Bio-Electromagnetics Laboratory, NTU, May. 2020 - Jul. 2022\hfill Taipei, Taiwan}
        { \small
        %Electrical system integration, mechanics design
        \begin{itemize}
            \item Designed an device to monitor the amount of bugs in farm fields with a custom PCB and rigorous mechanics.
            \item Created a git-based collaboration workflow for SolidWorks and a Notion-based BOM managment system to help me track quotes from vendors as well as the changes in design.
            \item Technical aspect involved industrial automation, IoT with Arduino (XBee), PCB design (Altium), SolidWorks and MySQL.
        \end{itemize}
        }
        
        
\end{document}

