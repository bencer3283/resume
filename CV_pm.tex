\documentclass[12pt]{article}
\usepackage[a4paper, total={18cm,27cm}]{geometry}
\usepackage{titlesec}
\usepackage{graphicx}
\usepackage{float}
\usepackage{multicol}
\setlength{\columnsep}{0.75cm}
\PassOptionsToPackage{hyphens}{url}\usepackage{hyperref}
%\usepackage{fancyhdr}
%\setlength{\headheight}{15pt}
%\pagestyle{empty}
%\fancyhead[L]{110-1 Bio-MEMS Fabrication Homework}
%\fancyhead[R]{鄭泊聲 b07611002}
\usepackage{amsmath}
\usepackage{latexsym}
\usepackage{multirow}
\graphicspath{ {./images/} }
%\usepackage[backend=biber, citestyle=numeric, bibstyle=numeric, sorting=none]{biblatex}
%\addbibresource{ref.bib}
\usepackage{mathspec}   %加這個就可以設定字體
\usepackage{xeCJK}       %讓中英文字體分開設置
\setCJKmainfont{Noto Serif CJK TC} %設定中文為系統上的字型
\newCJKfontfamily[chineseSans]\CJKsans{Noto Sans CJK TC}
\setmainfont{Sabon}
\setsansfont{Univers LT Std}%Univers LT Std 55}
\setmonofont{Roboto Mono}
\XeTeXlinebreaklocale "zh"             %這兩行一定要加,中文才能自動換行
\XeTeXlinebreakskip = 0pt plus 1pt     %這兩行一定要加,中文才能自動換行
\renewcommand{\baselinestretch}{1}
\renewcommand{\figurename}{圖}
\renewcommand{\tablename}{表}
\renewcommand{\abstractname}{摘要}
\renewcommand{\contentsname}{目錄}
\renewcommand{\listtablename}{表格目錄}
%\renewcommand*{\bibfont}{\footnotesize}
\titleformat*{\section}{\Large \bfseries}
\titleformat*{\subsection}{\large \sffamily}
\titleformat*{\subsubsection}{\bfseries}
%\setcounter{tocdepth}{2}

\begin{document}
%\pagenumbering{gobble}
    \section*{Po-Sheng Cheng \quad 鄭泊聲} 
    %{
    {\sffamily \href{https://bencer3283.github.io}{\underline{bencer3283.github.io/}}\quad bencer@outlook.com \quad +886-910-094-213}
    \newline
    %\section*{Education}
    \subsection*{National Taiwan University (NTU)} 
    {\sffamily
    \begin{itemize}
        \item Pursuing Master of Science, Graduate Insitute of Biomedical Electronics and Bioinformatics, College of EECS.
        \item Acquired Bechalor of Science in Bio-Mechatronics Engineering. Strong technical knowledge and interests of tech industry trend.
        \item Acqurred Bechalor of Arts in Economics. Skills for data science and finance.
    \end{itemize}
    IELTS score: 8.}

    \section*{Experiences
    % \footnote{\sffamily My coding experiences: Java for data structure, Matlab/R for statistics, C++/Kotlin for app development, Python for data processing on Raspberry Pi, Dart for UI, Labview for system integration and data visualization.}
    }
    % {\sffamily UI/UX, Econometrics, Machanics Design, Mechatronics, Electronics, Software Development. 
    % }
    
        \subsection*{Electro-mechanical Engineering Intern, Logitech}
        {\sffamily
        \begin{itemize}
            \item Feb. - Jun. 2022
            \item I proposed an innovative keyboard switch, then conducted a CI (Customer Insight) survey with 50 interviewees to understand its target audience and lastly designed three working prototype to demonstrate the technology.
            \item I gained strong familiarity with the NPI (New Product Introduction) process in the tech industry while collaborating with many departments (PM, EE, ID) in the company.
            %\item Besides statistics with R for the UX survey, my engineering capabilities including modeling with Creo, SLA/3DP prototyping, electromagnet and microcontroller circuits design and stepper motor control were involved.
        \end{itemize}
        }
        \subsection*{College Student Researcher, NTU}
        {\sffamily
        \begin{itemize}
            \item Jul. 2021 - Feb. 2022. A self-managed research project funded by Natinoal Science and Technology Council.
            \item I developed a PC-based software with C++, LabVIEW and Flutter to integrate a novel spectral mapping system named HSI. \href{https://github.com/HyperSpectral-Imaging}{\underline{GitHub repo of this project.}}
            %\item I developed a novel spectral mapping system named HSI that integrates several electrical/optical components with LabVIEW software development environment. %\newline \href{https://bencer3283.github.io/experiences/collegeStudentResearch/}{\underline{Details linked here.}}
            \item I won the following awards with this project: 
            \begin{itemize}
                \item \textbf{2021 Technology Innovation Award} by CCMS, NTU (NTD\$20,000)
                \item \textbf{College Student Research Creativity Award} by National Science and Technology Council of Taiwan (NTD\$20,000)
            \end{itemize}
        \end{itemize}}
        \subsection*{Project Lead, Bio-Electromagnetics Laboratory, NTU}
        {\sffamily
        \begin{itemize}
            \item May. 2020 - Jul. 2022
            \item I designed an IoT machine to monitor the amount of bugs in farm fields with inhouse-designed microcontroller PCB and mechanics.
            \item I managed a complex BOM of both mechanical and electrical components for the iterations of the device with quotes from different vendor candidates.
            \item Technical aspect involved automatic control, IoT with Arduino (XBee), PCB design (Altium), Python, SolidWorks, Raspberry Pi, MySQL.
        \end{itemize}
        }
        
    \section*{Awards and others}
    \subsection*{Championship, 2021 National Thesis Competition for College Students}
    {\sffamily
    \begin{itemize}
        \item I conducted a market survey with $\sim$700 samples and used regression analysis to understand how customer's preferences for online video streaming platform changed during the pandemic.
        \item We showed a surprising results that customers didn't find those platforms more appealing despite the pandemic forcing them to use those platforms more.
        This could partly explain the recent turmoils in the video streaming industry. 
        \item Awarded NTD\$30,000. \href{https://bencer3283.github.io/experiences/covidthesis/}{\underline{Link to the paper.}}
    \end{itemize}
    }
    
    \subsection*{Director of Academic Affairs, NTU Student Association of College of Bio-Res. and Agri.}
    {\sffamily \begin{itemize}
        \item Demonstrated outstanding communication skills while arranging several events with our industry partners.
    \end{itemize}}
    \subsection*{President, NTU Sunshiner}
    {\sffamily 
    \begin{itemize}
        \item Showed my leadership capabilities in a social service club providing free english classes for students in rural areas.
    \end{itemize} }
    

        % \subsection*{Self-directed project, Bio-Electromagnetics Laboratory, NTU}
        % {\sffamily
        % Electrical system integration, mechanics design
        % \begin{itemize}
        %     \item May. 2020 - Jul. 2022
        %     \item I designed a IoT machine to monitor the amount of bugs in farm fields with inhouse-designed controller board and mechanics.
        %     \item Technical aspect involved mechatronics, IoT with Arduino (XBee), PCB design (Altium), Python, SolidWorks, Raspberry Pi, MySQL.
        % \end{itemize}
        % }
        % \section*{Publication}
        % \subsection*{TeleSHift: Telexisting TUI for Physical Collaboration \& Interaction}
        % {\sffamily
        % {\footnotesize (forthcoming) Andrew Chen, Tzu-Ling Yang, Shu-Yan Cheng, \textbf{Po-Sheng Cheng}, Tzu-Han Lin, and Kaiyuan Lin. 2022. In Proceedings of the 2022 ACM International Joint Conference on Pervasive and Ubiquitous Computing (UbiComp/ISWC '22 Adjunct)%, September 11-15, 2022, Cambridge, United Kingdom. ACM, New York, NY, USA, 4 pages. 
        % , https://doi.org/10.48550/arXiv.2209.08362}
        % %\vspace*{4pt} \newline Human computer interaction
        % \begin{itemize}
        %     \item Apr. - Sep. 2022
        %     \item This work recieved the \textbf{Best Demo at Ubicomp/ISWC 2022}.
        %     \item In this work, a 3D tangible user interface (TUI) with telexisting communication framework for group-based collaboration is presented.
        %     \item I laid out the system architecture of the presented prototype in this work with ESP32 microcontroller, potentiometers, DC motors, etc. I also implemented the mechanical design and power supply circuit of the prototype.
        %     \item Additionally, my design for manufacturing improvements helped reducing assembling time of the prototype to only 1/5.
        % \end{itemize}
        % }
       
        % \subsection*{Golden Medalist, 19th Mobileheroes Award}
        % {\sffamily
        % \begin{itemize}
        %     \item Dec. 2020, Category of 5G innovative application, awarded NTD\$300,000 by Industrial Development Bureau of Taiwan.
        %     \item Our team \href{https://bencer3283.github.io/experiences/5G/}{\underline{ARGO}} developed an AR platform that utilizes advanced image-based spatial recognition algorithm.   
        % \end{itemize}
        % }
        
\end{document}

