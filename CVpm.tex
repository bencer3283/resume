\documentclass[12pt]{article}
\usepackage[a4paper, total={17cm,25cm}]{geometry}
\usepackage{titlesec}
\usepackage{graphicx}
\usepackage{float}
\usepackage{multicol}
\setlength{\columnsep}{1cm}
\PassOptionsToPackage{hyphens}{url}\usepackage{hyperref}
%\usepackage{fancyhdr}
%\setlength{\headheight}{15pt}
%\pagestyle{empty}
%\fancyhead[L]{110-1 Bio-MEMS Fabrication Homework}
%\fancyhead[R]{鄭泊聲 b07611002}
\usepackage{amsmath}
\usepackage{latexsym}
\usepackage{multirow}
\graphicspath{ {./images/} }
%\usepackage[backend=biber, citestyle=numeric, bibstyle=numeric, sorting=none]{biblatex}
%\addbibresource{ref.bib}
\usepackage{mathspec}   %加這個就可以設定字體
\usepackage{xeCJK}       %讓中英文字體分開設置
\setCJKmainfont{Noto Serif CJK TC} %設定中文為系統上的字型
\newCJKfontfamily[chineseSans]\CJKsans{Noto Sans CJK TC}
\setmainfont{Sabon}
\setsansfont{Univers LT Std}%Univers LT Std 55}
\setmonofont{Roboto Mono}
\XeTeXlinebreaklocale "zh"             %這兩行一定要加,中文才能自動換行
\XeTeXlinebreakskip = 0pt plus 1pt     %這兩行一定要加,中文才能自動換行
\renewcommand{\baselinestretch}{1}
\renewcommand{\figurename}{圖}
\renewcommand{\tablename}{表}
\renewcommand{\abstractname}{摘要}
\renewcommand{\contentsname}{目錄}
\renewcommand{\listtablename}{表格目錄}
%\renewcommand*{\bibfont}{\footnotesize}
\titleformat*{\section}{\Large \bfseries}
\titleformat*{\subsection}{\large \sffamily}
\titleformat*{\subsubsection}{\bfseries}
%\setcounter{tocdepth}{2}

\begin{document}
    \section*{Curriculum Vitae of Po-Sheng Cheng} 
    %\subsubsection*{for Design Science M.S., University of Michigan. (U-M ID: 91239502)}
    %{
    %\sffamily More details in personal website: \href{https://bencer3283.github.io}{\underline{bencer3283.github.io/}}}
    %\section*{Education}
    {\sffamily
    Bechalor of Science in \textbf{Bio-Mechatronics Engineering} and Bechalor of Arts in \textbf{Economics}, National Taiwan University (NTU).
    }
    
    \section*{Experiences
    % \footnote{\sffamily My coding experiences: Java for data structure, Matlab/R for statistics, C++/Kotlin for app development, Python for data processing on Raspberry Pi, Dart for UI, Labview for system integration and data visualization.}
    }
    % {\sffamily UI/UX, Econometrics, Machanics Design, Mechatronics, Electronics, Software Development. 
    % }
    %\begin{multicols*}{2}
        \subsection*{Electromechanical Engineering Intern, Logitech}
        {\sffamily
        UX survey, mechanics design
        \begin{itemize}
            \item Feb. - Jun. 2022
            \item I proposed a innovative keyboard switch, then conducted a UX survey with 50 interviewees to understand its target audience and lastly designed three working prototype to demonstrate the technology.
            \item Besides statistics with R for the UX survey, my engineering capabilities including modeling with Creo, SLA/3DP prototyping, electromagnet and microcontroller circuits design and stepper motor control were involved.
        \end{itemize}
        }
        \subsection*{College Student Researcher, NTU}
        {\sffamily
        Software development, system integration
        \begin{itemize}
            \item Jul. 2021 - Feb. 2022. A self-managed research project funded by Natinoal Science and Technology Council.
            \item I developed a novel spectral mapping system named HSI that integrates several electrical/optical components with LabVIEW software development environment. %\newline \href{https://bencer3283.github.io/experiences/collegeStudentResearch/}{\underline{Details linked here.}}
            \item By researching the readout sequence of the EMCCD and optimizing the algorithm, I reduced the scanning time by 26\%.
            \item Also involved Flutter UI/UX, C++, hyperspectral image processing, Electron Multiplying CCD and stepper motor control.
            \item I won the following awards with this project: 
            \begin{itemize}
                \item \textbf{2021 Technology Innovation Award} by CCMS, NTU (USD\$660)
                \item \textbf{College Student Research Creativity Award} by National Science and Technology Council of Taiwan (USD\$660)
            \end{itemize}
        \end{itemize}}
        \subsection*{Project Lead, Bio-Electromagnetics Laboratory, NTU}
        {\sffamily
        Electrical system integration, mechanics design
        \begin{itemize}
            \item May. 2020 - Jul. 2022
            \item I designed a IoT machine to monitor the amount of bugs in farm fields with inhouse-designed controller board and mechanics.
            \item Technical aspect involved mechatronics, IoT with Arduino (XBee), PCB design (Altium), Python, SolidWorks, Raspberry Pi, MySQL.
        \end{itemize}
        }
        % \section*{Competitions}
        % \subsection*{Championship, 2021 National Thesis Competition for College Students}
        % {\sffamily
        % {\footnotesize Covid-19's Impact on Online Video Streaming Platform from The Perspective of Consumer Preference. \textbf{Po-Sheng Cheng}, Ming-Chieh Chang, Hsuan-Yu Chou and others.}
        % \begin{itemize}
        %     \item Feb. - Apr. 2021
        %     \item Awarded USD\$1000.
        %     \item I used my econometrics skills(Chi-square test for independence, Logistic regression) to understand the relationship between customer's context and preferences.
        % \end{itemize}
        % }
        % \subsection*{Golden Medalist, 19th Mobileheroes Award}
        % {\sffamily
        % UI evaluation
        % \begin{itemize}
        %     \item Sep. - Dec. 2020
        %     \item Category of 5G innovative application, awarded USD\$10000 by Industrial Development Bureau of Taiwan.
        %     \item Our team ARGO has developed a AR platform that utilizes advanced image-based spatial recognition algorithm which enables real-time AR interactions on personal mobile devices. My main contribution was UI evaluation and design of the AR world for demo.   
        % \end{itemize}
        % }
        % \section*{Publication}
        % \subsection*{TeleSHift: Telexisting TUI for Physical Collaboration \& Interaction}
        % {\sffamily
        % {\footnotesize (forthcoming) Andrew Chen, Tzu-Ling Yang, Shu-Yan Cheng, \textbf{Po-Sheng Cheng}, Tzu-Han Lin, and Kaiyuan Lin. 2022. In Proceedings of the 2022 ACM International Joint Conference on Pervasive and Ubiquitous Computing (UbiComp/ISWC '22 Adjunct)%, September 11-15, 2022, Cambridge, United Kingdom. ACM, New York, NY, USA, 4 pages. 
        % , https://doi.org/10.48550/arXiv.2209.08362}
        % %\vspace*{4pt} \newline Human computer interaction
        % \begin{itemize}
        %     \item Apr. - Sep. 2022
        %     \item This work recieved the \textbf{Best Demo at Ubicomp/ISWC 2022}.
        %     \item In this work, a 3D tangible user interface (TUI) with telexisting communication framework for group-based collaboration is presented.
        %     \item I laid out the system architecture of the presented prototype in this work with ESP32 microcontroller, potentiometers, DC motors, etc. I also implemented the mechanical design and power supply circuit of the prototype.
        %     \item Additionally, my design for manufacturing improvements helped reducing assembling time of the prototype to only 1/5.
        % \end{itemize}
        % }
        
\end{document}

